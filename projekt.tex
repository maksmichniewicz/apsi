\documentclass[a4paper,titlepage,twoside,openright]{report} % twoside

\usepackage[utf8]{inputenc} 		% Kodowanie UTF8
\usepackage[MeX]{polski} 			% Wsparcie dla PL
\usepackage[pdftex]{graphicx} 		% Wsparcie dla obrazkow
\usepackage{fancyhdr}				% Nagłówki i stopki stron
									% Marginesy left = wewnetrzny margines
\usepackage{array}		
\usepackage[paper=a4paper,twoside,top=2.5cm,bottom=2.5cm,left=2.5cm,right=3cm,bindingoffset=1cm]{geometry}
 
\fancyhf{}
% Numer strony E-even, O-odd, L-left, R-right, C-center
\fancyhead[LE,RO]{\thepage} 
\fancyhead[RE]{\textit{\nouppercase{\leftmark}}}
\fancyhead[LO]{\textit{\nouppercase{\rightmark}}}

% Bez naglowka na poczatku rozdzialow
\fancypagestyle{plain}{ %
\fancyhf{} % remove everything
\renewcommand{\headrulewidth}{0pt} % remove lines as well
\renewcommand{\footrulewidth}{0pt}
}

\begin{document}	

	\title{System centralnego zarządzania wyświetlaczami reklamowymi}
	\author{Marcin Cieślikowski\\
		Marcin Kaczyński\\
		Marek Lewandowski\\
		Maksymilian Michniewicz\\}
	\date{\today}
	\maketitle
	
	\section*{Informacje o dokumencie}
	
	\subsubsection*{Metryka dokumentu}
		\begin{table}[h]
		\begin{tabular}{|p{2cm}|p{2cm}|p{2cm}|p{2cm}|m{3cm}|m{2cm}|}
			\hline \multicolumn{6}{|l|}{Metryka dokumentu} \\ 
			\hline
			\hline Projekt: & \multicolumn{3}{|m{6cm}|}{System centralnego zarządzania wyświetlaczami reklamowymi}  & Firma: & Politechnika Warszawska \\ 
			\hline Nazwa: & \multicolumn{5}{|l|}{Analiza systemu}  \\ 
			\hline Temat: & \multicolumn{5}{|l|}{Opis wymagań oraz projekt techniczny systemu}  \\ 
			\hline Autorzy: & \multicolumn{5}{|l|}{Marcin Cieślikowski, Marcin Kaczyński, Marek Lewandowski, Maksymilian Michniewicz}  \\ 
			\hline Plik: &  \multicolumn{5}{|l|}{\texttt{projekt.pdf}} \\ 
			\hline Nr. wersji: & \multicolumn{1}{|l|}{00.02} & Status: & Roboczy  & Data sporządzenia: & 11 grudnia 2011 \\ 
			\hline Streszczenie:  & \multicolumn{5}{|l|}{}  \\ % Jedno zdanie streszczenia 
			\hline Zatwierdził: & \multicolumn{3}{|l|}{} & Data ostatniej modyfikacji: & \today \\ 
			\hline 
		\end{tabular} 
		\end{table}

	\subsubsection*{Historia zmian}
		\begin{table}[h]
		\begin{tabular}{|m{1cm}|m{2cm}|m{3.5cm}|m{6.5cm}|}
			\hline
			\multicolumn{4}{|l|}{Historia zmian dokumentu} \\
			\hline
			\hline Wersja  & Data & Kto & Opis  \\ 
			\hline 00.01 & 01.12.2011 & Marcin Kaczyński & Stworzenie szablonu dokumentu w systemie \LaTeX \\ 
			\hline 00.02 & 09.12.2011 & Maksymilian Michniewicz & ... \\ 
			\hline 
		\end{tabular} 
		\end{table}
	\newpage

	
	%
	% SPIS TRESCI
	%
	\tableofcontents
	\newpage
	
	%
	% TRESC DOKUMENTU
	%

	\chapter{Wstęp}

		\section{Opis systemu}
			W systemie występuje centralny serwer z którym połączone są wyświetlacze reklamowe poprzez sieć internet. Dzięki temu wyświetlacze mogą być rozproszone po każdym obszarze z dostępem do internetu. Do serwera dostęp następuje poprzez przeglądarke WWW. Dzięki takiej architekturze systemu można nim zarządzać z każdego miejsca z dostępem do internetu.
			\subsection{Dostawca treści reklamowej}
				Dostawca treści ma możliwość zarządzaniem treścią reklamową wyświetlaną na wyświetlaczach. Tworząc nową treść może przetestować ją na testowym wyświetlaczu aby następnie wyświetlić ją na wszystkich wyświetlaczach lub wybranej grupie wyświetlaczy Może tworzyć kampanie, które będą miały określoną datę rozpoczęcia i zakończenia. Dostawców treści może być wielu oraz każdy może mieć ograniczony dostęp tylko do wybranych wyświetlaczy.
			\subsection{Obsługa serwisowa}
				W celu zapewnienia maksymalnej niezawodności systemu (każdy czas awarii systemu może kosztować dużą kwotę pieniędzy) system powinien być pod nieustanną obserwacją serwisową. System powinien udostępniać informacje na temat stanu wyświetlaczy takich jak:
				czy występuje połączenie wyświetlacza z serwerem,
 				w jakich godzinach wyświetlacz działał i był połączony
 				wystąpienie błędów 
				Serwisant może wykonać zdalnie test diagnostyczny, który stara się wykryć usterki sprzętowe np. dysku twardego. System umożliwia zdalną aktualizacje oprogramowania na wyświetlaczu oraz zmianę jego konfiguracji np. godziny pracy wyświetlacza. W przypadku awarii wyświetlacza system powinien powiadomić serwisanta odpowiedzialnego za dany terminal drogą mailową i SMS-ową. Serwisant może mieć dostęp tylko do wybranych terminali np. występujących w jego mieście lub województwie. 
			\subsection{Właściciel systemu}
				Właściciel ma dostęp do tych samych funkcji systemu co dostawca treści i serwisant ponadto ma możliwość zarządzania serwisantami i dostawcami treści. Może dodawać, usuwać im konta w systemie oraz zmieniać hasła i przydzielać im konkretne terminale lub grupy terminali.
		\section{Przeznaczenie systemu}

	\chapter{Analiza wymagań}
		\section{Wymagania funkcjonalne}
		    tekstu, więcej tekstu!!!tekstu, więcej tekstu!!!tekstu, więcej tekstu!!!tekstu, więcej tekstu!!!tekstu, więcej tekstu!!!
		\section{Wymagania niefunkcjonalne}
		    tekstu, więcej tekstu!!!tekstu, więcej tekstu!!!tekstu, więcej tekstu!!!tekstu, więcej tekstu!!!tekstu, więcej tekstu!!!
	% \cleardoublepage
	% \pagestyle{fancy}	
	
	\chapter{Specyfikacja przypadków użycia}	
	
		Tutaj będą przypadki użycia na poziomie ogólnym. Inicjowane są przez aktora, więc "nie wystają na zewnątrz". W zasadzie opisują użycie interfejsu. kominukację uzytkownik - system, od strony uzytkownika.\newline
		Diagramy aktywności mogą się tutaj przydać, na pewno diagramy typu Use Case nie są tutaj ważne.\newline
		\section{Aktorzy}
		{\bf Klient} - aktor główny, aktywny - użytkownik zalogowany.\newline
		{\bf Potencjalny klient} - aktor drugorzędny, aktywny - użytkownik internetu z zewnątrz.\newline
		{\bf Administrator systemu} - aktor drugorzędny, pasywny - pracownik badający system od strony technicznej - poprawnego działania.\newline
		{\bf Właściciel / analityk} - aktor główny, aktywny - pracownik robiący statystyki i zestawienia okresowe.\newline
		...coś jeszcze? wypisać...\newline
		możliwe, że jest jakiś aktor pasywny - wtedy trzeba go opisać i zapisać kiedy się uaktywnia (co go uaktywnia [tak chyba będzie działał administrator - przynajmniej w akcjach, gdzie ma reagować na niepożądane działanie systemu].)\newline
		aktorem prawdopodobnie będą także urządzenia - w naszym przypadku wyświetlacze, które będą aktywnie komunikować się z systemem (np wysyłając powiadomienia o błędzie).\newline
		\section{Systemowe przypadki użycia}
		Poniżej przedstawione zostały przypadki użycia opisujące interakcję obiektów zewnętrznych z systemem.\newline
		Opis ma charakter typu BlackBox. Implementacja wewnętrzna systemu jest pomijana na tym etapie. Opis przypadków jest typu Essential, czyli jest wolny od szczegółów technicznych i implementacyjnych.\newline

		\subsection{Identyfikacja przypadków użycia poprzez aktorów}
		\begin {center}
		\begin{tabular}{|p{4cm}|p{8cm}|}
		  \hline
		  {\bf Aktor} & {\bf Akcja} \\ \hline
		  \hline
		  Klient & Logowanie się do systemu \\ \cline{2-2}
		  & Wylogowanie się z systemu \\ \cline{2-2}
		  & Dodanie zamówienia \\ \cline{2-2}
		  & Przeglądanie dotychczasowych zamówień \\ \cline{2-2}
		  & Przeglądanie szczegółowe jednego zamówienia \\ \cline{2-2}
		  & Edycja zamówienia \\ \cline{2-2}
		  & Wycofanie zamówienia \\ \cline{2-2}
		  & Przeglądanie danych osobistych \\ \cline{2-2}
		  & Zmiana danych osobistych \\ \cline{2-2}
		  & Skasowanie konta ??? czy jest to możliwe? na jakich warunkach \\ \cline{2-2}
		  & ...co jeszcze? \\ \cline{2-2}
		  \hline
		  Potencjalny klient & Rejestracja w systemie \\ \cline{2-2}
		  \hline
		  Administrator systemu & Logowanie się do systemu na prawach administratora \\ \cline{2-2}
		  & Wylogowanie się z systemu \\ \cline{2-2}
		  & Monitorowanie pracy systemu \\ \cline{2-2}
		  & ...co jeszcze? \\ \cline{2-2}
		  \hline
		  Właściciel / analityk & Logowanie się do systemu na prawach pracownika \\ \cline{2-2}
		  & Wylogowanie się z systemu \\ \cline{2-2}
		  & Tworzenie zestawień okresowych \\ \cline{2-2}
		  & Tworzenie statystyk \\ \cline{2-2}
		  & ...co jeszcze? \\ \cline{2-2}
		  \hline
		  ?Kto jeszcze? & co jeszcze? \\ \cline{2-2}
		  \hline
		\end{tabular}
		\end{center}

		\subsection{Identyfikacja przypadków użycia poprzez zdarzenia}
		\begin {center}
		\begin{tabular}{|p{6cm}|p{6cm}|}
		  \hline
		  {\bf Zdarzenie} & {\bf Odpowiedź, udział aktorów} \\ \hline
		  \hline
		  Awaria systemu & System wysyła informację do administratora \\ \hline
		  Awaria wyświetlacza & System wysyła informację od administratora \\ \hline
		  Wpłata nie dotarła na określony czas & System wysyła przypomnienie \\ \hline
		  Wpłata nie została dokonana & System anuluje zamówienie \\ \hline
		  Klient kasuje swoje konto w systemie & System weryfikuje stan zamówień i płatności klienta \\ \hline
		  & ...co jeszcze? \\ \cline{2-2}
		  \hline
		\end{tabular}
		\end{center}
		\subsection{Identyfikacja przypadków użycia poprzez funkcje}
		\begin {center}
		\begin{tabular}{|p{6cm}|p{6cm}|}
		  \hline
		  {\bf Funkcje systemu} & {\bf Kontekst} \\ \hline
		  \hline
		  Dodawanie / usuwanie / edycja / wyswietlanie danych o użytkownikach & ... \\ \hline
		  Dodawanie / usuwanie / edycja / wyświetlanie danych o zamówieniach & ... \\ \hline
		  Weryfikacja zamówienia & Przy dodawaniu nowego zamówienia przez klienta \\ \hline
		  Zestawienie zamówienia z możliwościami systemu & Przy dodawaniu nowego zamówienia przez klienta \\ \hline
		  Weryfikacja zapłaty za zamówienie & ??? Przy dodawaniu nowego zamówienia przez klienta \\ \hline
		  Wprowadzenie zamówienia do systemu & Przy dodawaniu nowego zamówienia przez klienta \\ \hline
		  Tworzenie zestawień danych w systemie & W trakcie pracy analityka \\ \hline
		  & ...co jeszcze? \\ \cline{2-2}
		  \hline
		\end{tabular}
		\end{center}

		Mając coś takiego możemy weryfikować spójność, kompletność i niesprzeczność :)\newline
		...wypisać...\newline

		\section{Opis ogólny zidentyfikowanych przypadków użycia systemu}

		\begin {center}
		\begin{tabular}{|l|p{10cm}|}
		  \hline
		  \multicolumn{2}{|c|}{\bf Przypadek użycia - Dodanie reklamy}\\\hline
		  \hline
		  Aktor & Użytkownik \\ \hline
		  ID & TPU1 \\ \hline
		  Typ & Przebieg typowy \\ \hline
		  Opis & Użytkownik wprowadza nową reklamę i dane o reklamie i warunki na jakich chce ją wyświetlać. \\ \hline
		  Warunki początkowe & Użytkownik jest zalogowany \\ \hline
		  Warunki końcowe & ??? \\
		  \hline
		\end{tabular}
		\end{center}

		Pamiętać o:\newline
		Przebieg zdarzeń:\newline
		*typowy,\newline
		*alternatywne,\newline
		*wyjątki\newline
		*opis tekstowy, tabelaryczny,\newline
		*diagramy aktywności\newline

		\begin {center}
		{\bf Przebieg zdarzeń TPU1:}\newline
		\begin{tabular}{|p{6cm}|p{6cm}|}
		  \hline
		  \multicolumn{2}{|l|}{Inicjacja: Zalogowany użytkownik przegląda główne okno interfejsu}\\\hline
		  \hline
		  {\bf Aktor} & {\bf System }\\ \hline
		  \hline \hline
		  Użytkownik wybiera akcję (naciska przyscisk) `Dodaj zamówienie` & System wyświetla formularz \\ \hline
		  Użytkownik wypełnia formularz i wybiera akcję 'Prześlij' & System wyświetla informację o możliwościach zrealizowania zamówienia\\ \hline
		  Użytkownik zgadza się na warunki wybierając akcję 'Akceptuj' & System pobiera dane i wyświetla komunikat o wprowadzeniu zamówienia \\
		  \hline
		\end{tabular}
		\end{center}
		
\cleardoublepage
%\pagestyle{empty}

\chapter*{Słownik}
	\addcontentsline{toc}{chapter}{Słownik}
		
		\begin{description}
			\item[Pojęcie]
				Lorem ipsum dolor sit amet, consectetur adipisicing elit, sed do eiusmod tempor incididunt 
				ut labore et dolore magna aliqua. Ut enim ad 	minim veniam, quis nostrud exercitation 
				ullamco laboris nisi ut aliquip ex ea commodo consequat. Duis aute irure dolor in reprehenderit 
				in voluptate velit esse cillum dolore eu fugiat nulla pariatur. Excepteur sint occaecat 
				cupidatat non proident, sunt in culpa qui officia deserunt mollit anim id est laborum.				
			\item[Pojęcie] (ppp) 
				Lorem ipsum dolor sit amet, consectetur adipisicing elit, sed do eiusmod tempor incididunt 
				ut labore et dolore magna aliqua. Ut enim ad 	minim veniam, quis nostrud exercitation 
				ullamco laboris nisi ut aliquip ex ea commodo consequat. Duis aute irure dolor in reprehenderit 
				in voluptate velit esse cillum dolore eu fugiat nulla pariatur. Excepteur sint occaecat 
				cupidatat non proident, sunt in culpa qui officia deserunt mollit anim id est laborum.
		\end{description}
	
\end{document}
