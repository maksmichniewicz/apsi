%\documentclass[a4paper,titlepage,twoside,openright]{report} % twoside
\documentclass[10pt,a4paper,titlepage]{article} % twoside

\usepackage[utf8]{inputenc} 		% Kodowanie UTF8
\usepackage[MeX]{polski} 			% Wsparcie dla PL
\usepackage[pdftex]{graphicx} 		% Wsparcie dla obrazkow
\usepackage{fancyhdr}				% Nagłówki i stopki stron
									% Marginesy left = wewnetrzny margines
\usepackage{array}		
\usepackage[paper=a4paper,twoside,top=2.5cm,bottom=2.5cm,left=2.5cm,right=3cm,bindingoffset=1cm]{geometry}
 
\fancyhf{}
% Numer strony E-even, O-odd, L-left, R-right, C-center
\fancyhead[LE,RO]{\thepage} 
\fancyhead[RE]{\textit{\nouppercase{\leftmark}}}
\fancyhead[LO]{\textit{\nouppercase{\rightmark}}}

% Bez naglowka na poczatku rozdzialow
\fancypagestyle{plain}{ %
\fancyhf{} % remove everything
\renewcommand{\headrulewidth}{0pt} % remove lines as well
\renewcommand{\footrulewidth}{0pt}
}

\begin{document}	

	\title{System centralnego zarządzania wyświetlaczami reklamowymi}
	\author{Marcin Cieślikowski\\
		Marcin Kaczyński\\
		Marek Lewandowski\\
		Maksymilian Michniewicz\\}
	\date{\today}
	\maketitle
	
	\section*{Informacje o dokumencie}
	
	\subsubsection*{Metryka dokumentu}
		\begin{table}[h]
		\begin{tabular}{|p{2cm}|p{2cm}|p{2cm}|p{2cm}|m{3cm}|m{2cm}|}
			\hline \multicolumn{6}{|l|}{Metryka dokumentu} \\ 
			\hline
			\hline Projekt: & \multicolumn{3}{|m{6cm}|}{System centralnego zarządzania wyświetlaczami reklamowymi}  & Firma: & Politechnika Warszawska \\ 
			\hline Nazwa: & \multicolumn{5}{|l|}{Analiza systemu}  \\ 
			\hline Temat: & \multicolumn{5}{|l|}{Opis wymagań oraz projekt techniczny systemu}  \\ 
			\hline Autorzy: & \multicolumn{5}{|l|}{Marcin Cieślikowski, Marcin Kaczyński, Marek Lewandowski, Maksymilian Michniewicz}  \\ 
			\hline Plik: &  \multicolumn{5}{|l|}{\texttt{projekt.pdf}} \\ 
			\hline Nr. wersji: & \multicolumn{1}{|l|}{00.02} & Status: & Roboczy  & Data sporządzenia: & 11 grudnia 2011 \\ 
			\hline Streszczenie:  & \multicolumn{5}{|l|}{}  \\ % Jedno zdanie streszczenia 
			\hline Zatwierdził: & \multicolumn{3}{|l|}{} & Data ostatniej modyfikacji: & \today \\ 
			\hline 
		\end{tabular} 
		\end{table}

	\subsubsection*{Historia zmian}
		\begin{table}[h]
		\begin{tabular}{|m{1cm}|m{2cm}|m{3.5cm}|m{6.5cm}|}
			\hline
			\multicolumn{4}{|l|}{Historia zmian dokumentu} \\
			\hline
			\hline Wersja  & Data & Kto & Opis  \\ 
			\hline 00.01 & 01.12.2011 & Marcin Kaczyński & Stworzenie szablonu dokumentu w systemie \LaTeX \\ 
			\hline 00.02 & 09.12.2011 & Maksymilian Michniewicz & ... \\ 
			\hline 00.03 & 11.12.2011 & Marek Lewandowski & poprawki w wstępie \\ 
			\hline 00.04 & 13.12.2011 & Marek Lewandowski & uc rejestracja \\ 
			\hline 00.05 & 13.12.2011 & Marek Lewandowski & wstepe wymagania
			niefunkcjonalne
			\\
			\hline 
		\end{tabular} 
		\end{table}
	\newpage

	
	%
	% SPIS TRESCI
	%
	\tableofcontents
	\newpage
	
	%
	% TRESC DOKUMENTU
	%

	\section{Wstęp}

		Niniejszy dokument zawiera kompletną analizę systemu zarządzania treściami
		reklamowymi.\\ 
		
		Firma ABC dysponuje architekturą sprzętową, przygotowaną do wyświetlania
		reklam w autobusach, na elektronicznych bilbordach oraz na stronach
		internetowych. Analiza rynku wykazuje, że zarządzanie tak szeroką architekturą,
		przy przewidywanej liczbie dostawców treści reklamowych nie jest możliwe
		bez posiadania efektywnych narzędzi wspierajacych proces zarządzania i obsługi
		posiadanego sprzętu. Aby móc w pełni wykorzystać możliwości platformy sprzętowej
		konieczne jest stworzenie systemu informatycznego, który umożliwi gromadzenie i 
		przetwarzanie danych. Wobec powyższego zarząd firmy zdecydował o konieczności
		utworzenia systemu wspierającego proces zbierania zamówień na wyświetlanie
		treści reklamowych, zarządzanie procesem przetwarzania zebranych zamówień jak 
		również realizowanie zamówienia\\
		
		Podstawowym zadaniem systemu, jest wspomaganie procesu gromadzenia i obsługa
		treści reklamowych. System wspiera proces w szerokim zakresie interakcji z
		klientem.\\
		
		Zakres możliwości systemu jest funkcjonalnie kompletny, co w tym przypadku
		oznacza m. in.:\\
		
		\begin{itemize}
		\item zarządzanie danymi klientów,
		\item zarządzanie urządzeniami przeznaczonymi do wyświetlania treści,
		\item możliwość automatycznego generowania odpowiedzi na zapytania
		ofertowe,
		\item gromadzenie danych o historii transakcji
		\end{itemize}

		
		System stanowi wewnętrzny projekt firmy.
	
		\subsection{Opis systemu}
		Zarejestrowany w systemie dostawca treści reklamowych (zwany dalej użytkownikiem
		lub klientem) podpisuje z ABC umowę, na realizację usług zgodnie ze składanymi
		w systemie zamówieniami. Nie ma znaczenia czy klient planuje regularną
		współpracę z ABC, czy też chce zrealizować jednorazowe zamówienie. Osoba posiadająca
		konto w systemie, jest więc klientem ABC i może wykonywać operacje w ramach
		udostępnionej klientom części systemu. Z punktu widzenia ABC w zakresie obsługi
		klienta, system zarządza kontami klientów, wspiera proces przygotowania umowy wstępnej
		oraz zarządza zamówieniami klientów. System generuje dokumenty dla klientów, ale
		nie odpowiada za egzekwowanie wynikających z dokumentów zapisów. W szczególności
		system nie zajmuje się księgowaniem i windykowaniem zobowiązań klientów.
		Omawianymi zagadnieniami zajmują się kompetentne działy ABC i w związku z tym
		zagadnienia nie będą omawiane w dokumencie.\\
		
		Użytkownik przesyła do systemu zapytanie ofertowe opisujące jego wymagania w zakresie
		charakteru reklamy, wymagań dotyczących ilości emisji oraz preferowanego
		czasu emisji. System przetwarza zapytanie klienta i w odpowiedzi generuje
		raport zawierający propozycję realizacji zamówienia wraz z harmonogramem
		emisji opisanych przez klienta treści. System zarządza zamówieniami w
		taki sposób, żeby maksymalnie efektywnie wykorzystywać architekturę
		sprzętową firmy.\\

		Na podstawie przetwarzanych zamówień, system generuje odpowiednie zapisy
		dla urządzeń, dzięki którym określone urządzenie kolejkuje zadania do
		wykonania. Każde urządzenie komunikuje się z serwerem aby pobrać listę
		zadań do realizacji. Zamówienia zapisane w systemie są uznawane za pewne,
		tj. takie które klient zaakceptował, opłacił i wyraził wyraźną chęć ich realizacji.
		Zgodnie z odpowiednimi zapisami w umowie podpisanej przez klienta, takiego
		zamówienia nie można odwołać i otrzymać zwrotu opłaty. Można natomiast zrezygnować
		z emisji, ale opłata nie zostanie zwrócona.\\
		
		System jest zbudowany w oparciu o architekturę klient-serwer w ramach
		której użytkownik (klient) komunikuje się z aplikacją przez przeglądarkę
		www. Zastosowanie takiego modelu pozwala użytkownikowi na uzyskanie dostępu
		z dowolnego miejsca z dostępem do internetu, dzięki czemu użytkownik w każdej
		chwili może sprawdzić stan i przebieg swojego zamówienia jak również dokonać
		kolejnego zamówienia. Z drugiej pozwala uniknąć kosztownych wdrożeń i aktualizacji.
		Aplikacja użytkownika jest czytelna i prosta.\\

	\section{Analiza wymagań}
		\subsection{Wymagania funkcjonalne}
		    tekstu, więcej tekstu!!!tekstu, więcej tekstu!!!tekstu, więcej tekstu!!!tekstu, więcej tekstu!!!tekstu, więcej tekstu!!!
		\subsection{Wymagania niefunkcjonalne}
		   \begin{itemize}
		     \item System powinien zapewniać dostępność na poziomie 90\%
		     \item Wyświetlacz w razie awarii powinien być naprawiony w ciągu 24h
		     \item Restart systemu i powrót do pełnej funkcjonalności nie może
		     przekroczyć 3h
		     \item Główna część systemu powiniena zapewniać bezpieczeństwo klasy
		     Contract Business (niezła beka SPRAWDZIC CZY TO NIE SIE NIE POMYLILEM)
		     \begin{itemize}
		      \item Pozostałe informacje powinny być dostępne publicznie
		     \end{itemize}
		     \item System powinien pozwolić na obsługę do 100.000 klientów
		     jednocześnie
		     \item System powinien pozwolić na obsługę do 25.000 wyświetlaczy na
		     terenie Polski
		     \item System powinien przechowywać materiały reklamowe klientów z okresu
		     ostatnich 3 miesięcy. Po tym okresie nie musi ich archiwizować.
		     \item System powinien być obsługiwalnym przez jednego administratora.
		     \item System powinien wykonywać swoje funkcje związane z wyświetlaniem
		     reklam klinetów w 100\% poprawnie
		     \item System powinien być łatwo kontrolowalnym przez administratora i
		     serwisantów
		     \item System powinien działać poprawnie pomimo awarii części z
		     wyświetlaczy
		     \item System powinien być prosty w obsłudze i zrozumiały dla użytkowników
		     \item System powinien umożliwiać przetłumaczenie go na inny język jeśli
		     zapadnie taka decyzja w przyszłości
		     \item Dane klientów powinny być przechowywane zgodnie aktualną ustawą o
		     przechowywaniu danych osobowych
		     \item System powinien być dostępny z każdego komputera z dostępem do
		     internetu wyposażonego w przeglądarke www
		     \item System powinien być testowalny automatycznie
		   \end{itemize}
	
	% \cleardoublepage
	% \pagestyle{fancy}	
	
	\section{Specyfikacja przypadków użycia}	
	
		Tutaj będą przypadki użycia na poziomie ogólnym. Inicjowane są przez aktora, więc "nie wystają na zewnątrz". W zasadzie opisują użycie interfejsu. kominukację uzytkownik - system, od strony uzytkownika.\newline
		Diagramy aktywności mogą się tutaj przydać, na pewno diagramy typu Use Case nie są tutaj ważne.\newline
		\subsection{Aktorzy}
		{\bf Klient} - aktor główny, aktywny - użytkownik zalogowany.\newline
		{\bf Potencjalny klient} - aktor drugorzędny, aktywny - użytkownik internetu z zewnątrz.\newline
		{\bf Administrator systemu} - aktor drugorzędny, pasywny - pracownik badający system od strony technicznej - poprawnego działania.\newline
		{\bf Właściciel / analityk} - aktor główny, aktywny - pracownik robiący statystyki i zestawienia okresowe.\newline
		...coś jeszcze? wypisać...\newline
		możliwe, że jest jakiś aktor pasywny - wtedy trzeba go opisać i zapisać kiedy się uaktywnia (co go uaktywnia [tak chyba będzie działał administrator - przynajmniej w akcjach, gdzie ma reagować na niepożądane działanie systemu].)\newline
		aktorem prawdopodobnie będą także urządzenia - w naszym przypadku wyświetlacze, które będą aktywnie komunikować się z systemem (np wysyłając powiadomienia o błędzie).\newline
		\subsection{Systemowe przypadki użycia}
		Poniżej przedstawione zostały przypadki użycia opisujące interakcję obiektów zewnętrznych z systemem.\newline
		Opis ma charakter typu BlackBox. Implementacja wewnętrzna systemu jest pomijana na tym etapie. Opis przypadków jest typu Essential, czyli jest wolny od szczegółów technicznych i implementacyjnych.\newline

		\subsubsection{Identyfikacja przypadków użycia poprzez aktorów}
		\begin {center}
		\begin{tabular}{|p{4cm}|p{8cm}|}
		  \hline
		  {\bf Aktor} & {\bf Akcja} \\ \hline
		  \hline
		  Klient & Logowanie się do systemu \\ \cline{2-2}
		  & Wylogowanie się z systemu \\ \cline{2-2}
		  & Dodanie zamówienia \\ \cline{2-2}
		  & Przeglądanie dotychczasowych zamówień \\ \cline{2-2}
		  & Przeglądanie szczegółowe jednego zamówienia \\ \cline{2-2}
		  & Edycja zamówienia \\ \cline{2-2}
		  & Wycofanie zamówienia \\ \cline{2-2}
		  & Przeglądanie danych osobistych \\ \cline{2-2}
		  & Zmiana danych osobistych \\ \cline{2-2}
		  & Skasowanie konta ??? czy jest to możliwe? na jakich warunkach \\ \cline{2-2}
		  & ...co jeszcze? \\ \cline{2-2}
		  \hline
		  Potencjalny klient & Rejestracja w systemie \\ \cline{2-2}
		  \hline
		  Administrator systemu & Logowanie się do systemu na prawach administratora \\ \cline{2-2}
		  & Wylogowanie się z systemu \\ \cline{2-2}
		  & Monitorowanie pracy systemu \\ \cline{2-2}
		  & ...co jeszcze? \\ \cline{2-2}
		  \hline
		  Właściciel / analityk & Logowanie się do systemu na prawach pracownika \\ \cline{2-2}
		  & Wylogowanie się z systemu \\ \cline{2-2}
		  & Tworzenie zestawień okresowych \\ \cline{2-2}
		  & Tworzenie statystyk \\ \cline{2-2}
		  & ...co jeszcze? \\ \cline{2-2}
		  \hline
		  ?Kto jeszcze? & co jeszcze? \\ \cline{2-2}
		  \hline
		\end{tabular}
		\end{center}

		\subsubsection{Identyfikacja przypadków użycia poprzez zdarzenia}
		\begin {center}
		\begin{tabular}{|p{6cm}|p{6cm}|}
		  \hline
		  {\bf Zdarzenie} & {\bf Odpowiedź, udział aktorów} \\ \hline
		  \hline
		  Awaria systemu & System wysyła informację do administratora \\ \hline
		  Awaria wyświetlacza & System wysyła informację od administratora \\ \hline
		  Wpłata nie dotarła na określony czas & System wysyła przypomnienie \\ \hline
		  Wpłata nie została dokonana & System anuluje zamówienie \\ \hline
		  Klient kasuje swoje konto w systemie & System weryfikuje stan zamówień i płatności klienta \\ \hline
		  & ...co jeszcze? \\ \cline{2-2}
		  \hline
		\end{tabular}
		\end{center}
		\subsubsection{Identyfikacja przypadków użycia poprzez funkcje}
		\begin {center}
		\begin{tabular}{|p{6cm}|p{6cm}|}
		  \hline
		  {\bf Funkcje systemu} & {\bf Kontekst} \\ \hline
		  \hline
		  Dodawanie / usuwanie / edycja / wyswietlanie danych o użytkownikach & ... \\ \hline
		  Dodawanie / usuwanie / edycja / wyświetlanie danych o zamówieniach & ... \\ \hline
		  Weryfikacja zamówienia & Przy dodawaniu nowego zamówienia przez klienta \\ \hline
		  Zestawienie zamówienia z możliwościami systemu & Przy dodawaniu nowego zamówienia przez klienta \\ \hline
		  Weryfikacja zapłaty za zamówienie & ??? Przy dodawaniu nowego zamówienia przez klienta \\ \hline
		  Wprowadzenie zamówienia do systemu & Przy dodawaniu nowego zamówienia przez klienta \\ \hline
		  Tworzenie zestawień danych w systemie & W trakcie pracy analityka \\ \hline
		  & ...co jeszcze? \\ \cline{2-2}
		  \hline
		\end{tabular}
		\end{center}

		Mając coś takiego możemy weryfikować spójność, kompletność i niesprzeczność :)\newline
		...wypisać...\newline

		\subsection{Opis ogólny zidentyfikowanych przypadków użycia systemu}

		\begin {center}
		\begin{tabular}{|l|p{10cm}|}
		  \hline
		  \multicolumn{2}{|c|}{\bf Przypadek użycia - Dodanie reklamy}\\\hline
		  \hline
		  Aktor & Użytkownik \\ \hline
		  ID & TPU1 \\ \hline
		  Typ & Przebieg typowy \\ \hline
		  Opis & Użytkownik wprowadza nową reklamę i dane o reklamie i warunki na jakich chce ją wyświetlać. \\ \hline
		  Warunki początkowe & Użytkownik jest zalogowany \\ \hline
		  Wynik końcowy & ??? \\
		  \hline
		\end{tabular}
		\end{center}
		
		\begin {center}
		\begin{tabular}{|l|p{10cm}|}
		  \hline
		  \multicolumn{2}{|c|}{\bf Przypadek użycia - Rejestracja użytkownika}\\\hline
		  \hline
		  Aktor & Użytkownik \\ \hline
		  ID & TPU2 \\ \hline
		  Typ & Przebieg typowy \\ \hline
		  Opis & Użytkownik chce się zarejestrować w systemie. Pierwszym etapem
		  rejestracji jest wypełnienie formularza na stronie. Następnym etapem jest podpisanie umowy przez
		  zainteresowanego, zeskanowanie jej i odesłanie na podany adres w celu
		  ukończenia rejestracji. \\
		  \hline Warunki początkowe & Brak \\ \hline
		  Wynik końcowy & użytkownik zarejestrowany w systemie \\
		  \hline
		\end{tabular}
		\end{center}

		Pamiętać o:\newline
		Przebieg zdarzeń:\newline
		*typowy,\newline
		*alternatywne,\newline
		*wyjątki\newline
		*opis tekstowy, tabelaryczny,\newline
		*diagramy aktywności\newline

		\begin {center}
		{\bf Przebieg zdarzeń TPU1:}\newline
		\begin{tabular}{|p{6cm}|p{6cm}|}
		  \hline
		  \multicolumn{2}{|l|}{Inicjacja: Zalogowany użytkownik przegląda główne okno interfejsu}\\\hline
		  \hline
		  {\bf Aktor} & {\bf System }\\ \hline
		  \hline \hline
		  Użytkownik wybiera akcję (naciska przyscisk) `Dodaj zamówienie` & System wyświetla formularz \\ \hline
		  Użytkownik wypełnia formularz i wybiera akcję 'Prześlij' & System wyświetla informację o możliwościach zrealizowania zamówienia\\ \hline
		  Użytkownik zgadza się na warunki wybierając akcję 'Akceptuj' & System pobiera dane i wyświetla komunikat o wprowadzeniu zamówienia \\
		  \hline
		\end{tabular}
		\end{center}
		
		\begin {center}
		{\bf Przebieg zdarzeń TPU2:}\newline
		\begin{tabular}{|p{6cm}|p{6cm}|}
		  \hline
		  \multicolumn{2}{|l|}{Inicjacja: Klient przegląda stronę główną}\\\hline
		  \hline
		  {\bf Aktor} & {\bf System }\\ \hline
		  \hline \hline
		  Klient wybiera akcję (naciska przyscisk) `Zarejestruj` & System
		  wyświetla formularz do rejestracji \\ \hline Klient wypełnia formularz i
		  wybiera akcję 'Prześlij' & System sprawdza poprawność danych \\ \hline
		  & System wysyła umowę na mail podany przez klienta\\ \hline
		  Klient podpisuje umowę, skanuje ją i odsyła & System zawiadamia menadżera
		  o nowym kliencie.\\
		  \hline
		\end{tabular}
		\end{center}
		
		\begin{description}
			\item[Przebieg zdarzeń TPU2:]
			\item[Inicjacja:] Klient przegląda stronę główną 
			\item[Aktorzy:]  Klient, System, Menadżer
		\end{description}
		
		{\bf Przebieg zdarzeń TPU2:}\newline
		{\bf Inicjacja:} Klient przegląda stronę główną \newline
		{\bf Aktorzy:} Klient, System, Menadżer 
			\begin{enumerate}
			  \item Klient wybiera akcję (naciska przycisk) 'Zarejestruj'
			  \item System wyświetla formularz do rejestracji
			  \item Klient wypełnia formularz i wybiera akcję `Prześlij`
			  \item System sprawdza poprawność danych
			  \item System wysyła uzupełnioną umowę na mail podany przez klienta
			  \item Klient drukuje umowę, podpisuje, skanuje ją i odsyła
			  \item System zawiadamia menadżera o podpisanej umowie
			  \item Menadżer zatwierdza umowę z klientem
			  \item System wysyła maila do klienta o pomyślnej rejestracji
			\end{enumerate}
			
			
		
		{\bf Przebieg alternatywny 1: Niepoprawne dane} \newline
			1. - 4. tak jak w scenariuszu główwym
			\begin{enumerate}
			  \setcounter{enumi}{4}
			  \item System wykrył niezgodność w danych i informuje o tym klienta
			  \item Klient poprawia dane i wybiera akcję `Prześlij'
			  \item Tak jak w scenariuszu głównym od 4.
			\end{enumerate}
			
	   {\bf Przebieg alternatywny 2: Umowa nie może zostać podpisana} \newline
			1. - 7. tak jak w scenariuszu głównym
			\begin{enumerate}
			  \setcounter{enumi}{7}
			  \item Menadżer nie może podpisać umowy (powod np. jakas reguła biznesowa
			  TO DO USTALENIA)
			  \item System wysyła maila do klienta o zaistniałej sytuacji
			\end{enumerate}
			
		{\bf Przebieg alternatywny 3: Klient nie odesłał umowy} \newline
			1. - 5. tak jak w scenariuszu głównym
			\begin{enumerate}
			  \setcounter{enumi}{5}
			  \item Przekroczony został czas oczekiwania na umowę (tutaj na pewno jakas
			  reguła biznesowa)
			  \item System odrzuca wniosek o rejestrację.
			\end{enumerate}
		
\cleardoublepage
%\pagestyle{empty}

\section*{Słownik}
	\addcontentsline{toc}{chapter}{Słownik}
		
		\begin{description}
			\item[Pojęcie]
				Lorem ipsum dolor sit amet, consectetur adipisicing elit, sed do eiusmod tempor incididunt 
				ut labore et dolore magna aliqua. Ut enim ad 	minim veniam, quis nostrud exercitation 
				ullamco laboris nisi ut aliquip ex ea commodo consequat. Duis aute irure dolor in reprehenderit 
				in voluptate velit esse cillum dolore eu fugiat nulla pariatur. Excepteur sint occaecat 
				cupidatat non proident, sunt in culpa qui officia deserunt mollit anim id est laborum.				
			\item[Pojęcie] (ppp) 
				Lorem ipsum dolor sit amet, consectetur adipisicing elit, sed do eiusmod tempor incididunt 
				ut labore et dolore magna aliqua. Ut enim ad 	minim veniam, quis nostrud exercitation 
				ullamco laboris nisi ut aliquip ex ea commodo consequat. Duis aute irure dolor in reprehenderit 
				in voluptate velit esse cillum dolore eu fugiat nulla pariatur. Excepteur sint occaecat 
				cupidatat non proident, sunt in culpa qui officia deserunt mollit anim id est laborum.
			\item[Test] Nawet nie pytajcie po co to dodałem ;)
		\end{description}
	
\end{document}
